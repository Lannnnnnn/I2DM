\chapter{Predicates and Connectives}
\begin{wrapfigure}{l}{4cm}
  \begin{center}
    \vskip -0.5cm
    \qrset{link, height=3cm}
    \qrcode{https://youtu.be/0unvlq2OTaE}
    \vskip 0.25cm
    \href{https://youtu.be/0unvlq2OTaE}{youtu.be/0unvlq2OTaE}
  \end{center}
  \vskip -0.5cm
\end{wrapfigure}

\section{Propositions and Predicates}
In the previous chapters we used the word ``statement'' without any even
relatively formal definition what it means. In this chapter we are going to
give a semi-formal definition and discuss how to create complicated statements
from simple statements.

It is difficult to give a formal definition what a mathematical statement is,
hence, we are not going to do it in this book. The goal of this section is to
enable the reader to recognize mathematical statements.

A \textit{proposition} or a mathematical statement is a declarative sentence
which is either true or false but not both. Consider the following list of
sentences.
\begin{enumerate}
  \item $2 \times 2 = 4$
  \item $\pi = 4$
  \item $n$ is even
  \item 32 is special
  \item The square of any odd number is odd.
  \item The sum of any even number and one is prime.
\end{enumerate}
Of those first two are propositions; note that this says nothing whether they
are true or not. Actually, the first is true and the second is false.
However, the third sentence became a proposition only when the value
of $n$ is fixed and the fours is not a proposition. Finally, the last
two are proposition.

Third statement is somewhat special, there is a simple way to make it a
proposition, one just need to fix the value of the variables. Such sentences
are called predicates and the variables that need to me specified are called
free variables of these predicates.

Note that the fourth sentence is also interesting, since if we define what it
means to be special, the phrase became a proposition.
Mathematicians are tend to do such things and to give mathematical meanings to
everyday words.

\section{Connectives}

Mathematicians often need to decide whether a given proposition is true or
false. Many statements are complicated and constructed from simpler statements
using \textit{logical connectives}. For example we may consider the following
statements:
\begin{enumerate}
  \item $3 > 4$ and $1 < 1$;
  \item $1 \times 2 = 5$ or $6 > 1$.
\end{enumerate}

\paragraph{Logical connective ``OR''.}
The second statement is an example of usage of this connective. The statement
``P or Q'' is true if and only if at least one of P and Q is true. We may
define the connective using the truth table of it.
\begin{center}
  \begin{tabular}{c | c | c}
    P & Q & P or Q \\
    \hline
    T & T & T \\
    T & F & T \\
    F & T & T \\
    F & F & F
  \end{tabular}
\end{center}

The or connective is also called \textit{disjunction} and the disjunction of $P$
and $Q$ is often dented as $P \lor Q$.

\begin{warning}
  Note that in everyday speech ``or'' is often used in the exclusive case, like
  in the sentence ``we need to decide whether it is an insect or a spider''.
  In this case the precise meaning of ``or'' is made clear by the context.
  However, mathematical language should be formal, hence, we always use ``or''
  inclusively.
\end{warning}

\paragraph{Logical connective ``AND''.}
The first statement is an example of this connective. The statement ``P or Q''
is true if and only if both P and Q are true. We may define the
connective using the truth table of it.
\begin{center}
  \begin{tabular}{c | c | c}
    P & Q & P and Q \\
    \hline
    T & T & T \\
    T & F & F \\
    F & T & F \\
    F & F & F
  \end{tabular}
\end{center}

The or connective is also called \textit{conjunction} and the conjunction of
$P$ and $Q$ is often dented as $P \land Q$.

\paragraph{Logical connective ``NOT''.}
The last connective is called \textit{negation} and examples of usage of it are
the following:
\begin{enumerate}
  \item 5 is not greater than 8;
  \item Does not exist an integer $n$ such that $n^2 = 2$.
\end{enumerate}

Note that it is not straightforward where to put the negation in these
sentences.
\section*{End of The Chapter Exercises}
\begin{exercises}
  \exerciseitem Construct truth tables for the statements
    \begin{itemize}
      \item not ($P$ and $Q$);
      \item (not $P$) or (not $Q$);
      \item $P$ and (not $Q$);
      \item (not $P$) or $Q$;
    \end{itemize}
  \exerciseitem Consider the statement ``All gnomes like cookies''. Which of
    the following statements is the negation of the above statement?
    \begin{itemize}
      \item All gnomes hate cookies.
      \item All gnomes do not like cookies.
      \item Some gnome do not like cookies.
      \item Some gnome hate cookies.
      \item All creatures who like cookies are gnomes.
      \item All creatures who do not like cookies are not gnomes.
    \end{itemize}
  \exerciseitem Using truth tables show that the following statements are
    equivalent:
    \begin{itemize}
      \item $P \implies Q$,
      \item $(P \lor Q) \iff Q$
        ($A \implies B$ is the same as $(A \implies B) \land (B \implies A)$),
      \item $(P \land Q) \iff P$
    \end{itemize}
  \exerciseitem Prove that three connectives ``or'', ``and'', and ``not'' can
    all be written in terms of the single connective ``notand'' where ``$P$
    notand $Q$'' is interpreted as ``not ($P$ and $Q$)''.
\end{exercises}
